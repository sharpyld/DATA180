% Options for packages loaded elsewhere
\PassOptionsToPackage{unicode}{hyperref}
\PassOptionsToPackage{hyphens}{url}
%
\documentclass[
]{article}
\usepackage{amsmath,amssymb}
\usepackage{lmodern}
\usepackage{iftex}
\ifPDFTeX
  \usepackage[T1]{fontenc}
  \usepackage[utf8]{inputenc}
  \usepackage{textcomp} % provide euro and other symbols
\else % if luatex or xetex
  \usepackage{unicode-math}
  \defaultfontfeatures{Scale=MatchLowercase}
  \defaultfontfeatures[\rmfamily]{Ligatures=TeX,Scale=1}
\fi
% Use upquote if available, for straight quotes in verbatim environments
\IfFileExists{upquote.sty}{\usepackage{upquote}}{}
\IfFileExists{microtype.sty}{% use microtype if available
  \usepackage[]{microtype}
  \UseMicrotypeSet[protrusion]{basicmath} % disable protrusion for tt fonts
}{}
\makeatletter
\@ifundefined{KOMAClassName}{% if non-KOMA class
  \IfFileExists{parskip.sty}{%
    \usepackage{parskip}
  }{% else
    \setlength{\parindent}{0pt}
    \setlength{\parskip}{6pt plus 2pt minus 1pt}}
}{% if KOMA class
  \KOMAoptions{parskip=half}}
\makeatother
\usepackage{xcolor}
\usepackage[margin=1in]{geometry}
\usepackage{color}
\usepackage{fancyvrb}
\newcommand{\VerbBar}{|}
\newcommand{\VERB}{\Verb[commandchars=\\\{\}]}
\DefineVerbatimEnvironment{Highlighting}{Verbatim}{commandchars=\\\{\}}
% Add ',fontsize=\small' for more characters per line
\usepackage{framed}
\definecolor{shadecolor}{RGB}{248,248,248}
\newenvironment{Shaded}{\begin{snugshade}}{\end{snugshade}}
\newcommand{\AlertTok}[1]{\textcolor[rgb]{0.94,0.16,0.16}{#1}}
\newcommand{\AnnotationTok}[1]{\textcolor[rgb]{0.56,0.35,0.01}{\textbf{\textit{#1}}}}
\newcommand{\AttributeTok}[1]{\textcolor[rgb]{0.77,0.63,0.00}{#1}}
\newcommand{\BaseNTok}[1]{\textcolor[rgb]{0.00,0.00,0.81}{#1}}
\newcommand{\BuiltInTok}[1]{#1}
\newcommand{\CharTok}[1]{\textcolor[rgb]{0.31,0.60,0.02}{#1}}
\newcommand{\CommentTok}[1]{\textcolor[rgb]{0.56,0.35,0.01}{\textit{#1}}}
\newcommand{\CommentVarTok}[1]{\textcolor[rgb]{0.56,0.35,0.01}{\textbf{\textit{#1}}}}
\newcommand{\ConstantTok}[1]{\textcolor[rgb]{0.00,0.00,0.00}{#1}}
\newcommand{\ControlFlowTok}[1]{\textcolor[rgb]{0.13,0.29,0.53}{\textbf{#1}}}
\newcommand{\DataTypeTok}[1]{\textcolor[rgb]{0.13,0.29,0.53}{#1}}
\newcommand{\DecValTok}[1]{\textcolor[rgb]{0.00,0.00,0.81}{#1}}
\newcommand{\DocumentationTok}[1]{\textcolor[rgb]{0.56,0.35,0.01}{\textbf{\textit{#1}}}}
\newcommand{\ErrorTok}[1]{\textcolor[rgb]{0.64,0.00,0.00}{\textbf{#1}}}
\newcommand{\ExtensionTok}[1]{#1}
\newcommand{\FloatTok}[1]{\textcolor[rgb]{0.00,0.00,0.81}{#1}}
\newcommand{\FunctionTok}[1]{\textcolor[rgb]{0.00,0.00,0.00}{#1}}
\newcommand{\ImportTok}[1]{#1}
\newcommand{\InformationTok}[1]{\textcolor[rgb]{0.56,0.35,0.01}{\textbf{\textit{#1}}}}
\newcommand{\KeywordTok}[1]{\textcolor[rgb]{0.13,0.29,0.53}{\textbf{#1}}}
\newcommand{\NormalTok}[1]{#1}
\newcommand{\OperatorTok}[1]{\textcolor[rgb]{0.81,0.36,0.00}{\textbf{#1}}}
\newcommand{\OtherTok}[1]{\textcolor[rgb]{0.56,0.35,0.01}{#1}}
\newcommand{\PreprocessorTok}[1]{\textcolor[rgb]{0.56,0.35,0.01}{\textit{#1}}}
\newcommand{\RegionMarkerTok}[1]{#1}
\newcommand{\SpecialCharTok}[1]{\textcolor[rgb]{0.00,0.00,0.00}{#1}}
\newcommand{\SpecialStringTok}[1]{\textcolor[rgb]{0.31,0.60,0.02}{#1}}
\newcommand{\StringTok}[1]{\textcolor[rgb]{0.31,0.60,0.02}{#1}}
\newcommand{\VariableTok}[1]{\textcolor[rgb]{0.00,0.00,0.00}{#1}}
\newcommand{\VerbatimStringTok}[1]{\textcolor[rgb]{0.31,0.60,0.02}{#1}}
\newcommand{\WarningTok}[1]{\textcolor[rgb]{0.56,0.35,0.01}{\textbf{\textit{#1}}}}
\usepackage{graphicx}
\makeatletter
\def\maxwidth{\ifdim\Gin@nat@width>\linewidth\linewidth\else\Gin@nat@width\fi}
\def\maxheight{\ifdim\Gin@nat@height>\textheight\textheight\else\Gin@nat@height\fi}
\makeatother
% Scale images if necessary, so that they will not overflow the page
% margins by default, and it is still possible to overwrite the defaults
% using explicit options in \includegraphics[width, height, ...]{}
\setkeys{Gin}{width=\maxwidth,height=\maxheight,keepaspectratio}
% Set default figure placement to htbp
\makeatletter
\def\fps@figure{htbp}
\makeatother
\setlength{\emergencystretch}{3em} % prevent overfull lines
\providecommand{\tightlist}{%
  \setlength{\itemsep}{0pt}\setlength{\parskip}{0pt}}
\setcounter{secnumdepth}{-\maxdimen} % remove section numbering
\ifLuaTeX
  \usepackage{selnolig}  % disable illegal ligatures
\fi
\IfFileExists{bookmark.sty}{\usepackage{bookmark}}{\usepackage{hyperref}}
\IfFileExists{xurl.sty}{\usepackage{xurl}}{} % add URL line breaks if available
\urlstyle{same} % disable monospaced font for URLs
\hypersetup{
  pdftitle={Lecture 4 cont: ggplot2},
  hidelinks,
  pdfcreator={LaTeX via pandoc}}

\title{Lecture 4 cont: ggplot2}
\author{}
\date{\vspace{-2.5em}}

\begin{document}
\maketitle

\hypertarget{ggplot2-r-package-for-data-visualization}{%
\section{ggplot2 R package for Data
Visualization}\label{ggplot2-r-package-for-data-visualization}}

\begin{itemize}
\tightlist
\item
  ggplot is a function in the
  \href{https://cran.r-project.org/web/packages/ggplot2/index.html}{ggplot2}
  package and is based on \textbf{The Grammar of Graphics} by Leland
  Wilkinson, and the
  \href{https://cran.r-project.org/web/packages/lattice/index.html}{lattice}
  package
\item
  ggplot is designed to work in a \emph{layered fashion}, starting with
  a layer showing the raw data then adding layers of annotation and
  statistical summaries
\end{itemize}

\hypertarget{ggplot-function}{%
\subsection{ggplot() function}\label{ggplot-function}}

Let's look at an example \textbf{diamonds} data that comes with the
\href{https://cran.r-project.org/web/packages/ggplot2/index.html}{ggplot2}
package. But first, let's load the the package to our session using the
\textbf{library()} function. If you have not yet installed the ggplot2
package, you should do this first (you only have to do this once).
Recall the magic command is: install.packages(``\,``)

Some information about the \textbf{diamonds} dataset :

\begin{itemize}
\tightlist
\item
  \textasciitilde54,000 round diamonds from
  \url{http://www.diamondse.info/}
\item
  Variables:

  \begin{itemize}
  \tightlist
  \item
    carat, colour, clarity, cut
  \item
    total depth, table, depth, width, height
  \item
    price
  \end{itemize}
\item
  A question of interest: What is the relationship between \emph{carat}
  and \emph{price}, and how does it depend on other factors?
\end{itemize}

\hypertarget{data-on-diamonds}{%
\subsection{Data on diamonds}\label{data-on-diamonds}}

Load the \textbf{diamonds} data set, get the dimensions, and look at the
first few lines

\begin{Shaded}
\begin{Highlighting}[]
\FunctionTok{data}\NormalTok{(diamonds, }\AttributeTok{package=}\StringTok{"ggplot2"}\NormalTok{)}
\FunctionTok{dim}\NormalTok{(diamonds)}
\CommentTok{\# [1] 53940    10}
\FunctionTok{head}\NormalTok{(diamonds)}
\CommentTok{\# \# A tibble: 6 x 10}
\CommentTok{\#   carat cut       color clarity depth table price     x     y     z}
\CommentTok{\#   \textless{}dbl\textgreater{} \textless{}ord\textgreater{}     \textless{}ord\textgreater{} \textless{}ord\textgreater{}   \textless{}dbl\textgreater{} \textless{}dbl\textgreater{} \textless{}int\textgreater{} \textless{}dbl\textgreater{} \textless{}dbl\textgreater{} \textless{}dbl\textgreater{}}
\CommentTok{\# 1  0.23 Ideal     E     SI2      61.5    55   326  3.95  3.98  2.43}
\CommentTok{\# 2  0.21 Premium   E     SI1      59.8    61   326  3.89  3.84  2.31}
\CommentTok{\# 3  0.23 Good      E     VS1      56.9    65   327  4.05  4.07  2.31}
\CommentTok{\# 4  0.29 Premium   I     VS2      62.4    58   334  4.2   4.23  2.63}
\CommentTok{\# 5  0.31 Good      J     SI2      63.3    58   335  4.34  4.35  2.75}
\CommentTok{\# 6  0.24 Very Good J     VVS2     62.8    57   336  3.94  3.96  2.48}
\end{Highlighting}
\end{Shaded}

\hypertarget{refresher-default-r-plot-of-price-versus-carat}{%
\subsection{Refresher: Default R plot of price versus
carat}\label{refresher-default-r-plot-of-price-versus-carat}}

Recall using the \emph{default settings} with the \textbf{plot()}
function

\begin{Shaded}
\begin{Highlighting}[]
\FunctionTok{plot}\NormalTok{(diamonds}\SpecialCharTok{$}\NormalTok{carat,diamonds}\SpecialCharTok{$}\NormalTok{price) }\CommentTok{\# x{-}variable first in this notation}
\end{Highlighting}
\end{Shaded}

\begin{figure}

{\centering \includegraphics{ggplot_class_practice_files/figure-latex/unnamed-chunk-4-1} 

}

\caption{Here is a caption.}\label{fig:unnamed-chunk-4-1}
\end{figure}

\begin{Shaded}
\begin{Highlighting}[]
\CommentTok{\# or}
\FunctionTok{plot}\NormalTok{(price}\SpecialCharTok{\textasciitilde{}}\NormalTok{carat, }\AttributeTok{data=}\NormalTok{diamonds) }\CommentTok{\# an alternative way: this is y against x}
\end{Highlighting}
\end{Shaded}

\begin{figure}

{\centering \includegraphics{ggplot_class_practice_files/figure-latex/unnamed-chunk-4-2} 

}

\caption{Here is a caption.}\label{fig:unnamed-chunk-4-2}
\end{figure}

Note: If you want to change the figure size and add a figure caption in
Rmd, you can specify the fig.height, fig.width, fig.cap options inside
curly brackets at the beginning of your R code chunk. See the full set
of options \href{https://rpubs.com/Lingling912/870659}{here} Bonus: If
you don't want to set the size each time you generate a plot, you can
insert the options, e.g., fig\_width fig\_height in your YAML chunk at
the beginning of your Rmd file.

\hypertarget{default-ggplot-price-versus-diamonds}{%
\subsection{Default ggplot price versus
diamonds}\label{default-ggplot-price-versus-diamonds}}

Here comes ggplot! Using the \emph{default settings} in
\textbf{ggplot()}

\begin{Shaded}
\begin{Highlighting}[]
\FunctionTok{theme\_set}\NormalTok{(}\FunctionTok{theme\_bw}\NormalTok{()) }\DocumentationTok{\#\#set b+w color}
\FunctionTok{library}\NormalTok{(ggplot2) }\DocumentationTok{\#\# source of the plot}
\FunctionTok{ggplot}\NormalTok{(diamonds, }\FunctionTok{aes}\NormalTok{(}\AttributeTok{x=}\NormalTok{carat,}\AttributeTok{y=}\NormalTok{price)) }\SpecialCharTok{+} 
                          \FunctionTok{geom\_point}\NormalTok{() }\SpecialCharTok{+} 
                          \FunctionTok{labs}\NormalTok{(}\AttributeTok{y =} \StringTok{\textquotesingle{}price $\textquotesingle{}}\NormalTok{, }\AttributeTok{x =} \StringTok{\textquotesingle{}carat value\textquotesingle{}}\NormalTok{) }\DocumentationTok{\#\#details}
\end{Highlighting}
\end{Shaded}

\begin{center}\includegraphics{ggplot_class_practice_files/figure-latex/unnamed-chunk-5-1} \end{center}

\hypertarget{at-first-glance}{%
\subsection{At first glance}\label{at-first-glance}}

\begin{itemize}
\tightlist
\item
  The \emph{default} option in \textbf{ggplot} looks nicer!
\item
  \textbf{ggplot}'s syntax looks weird, especially if you're not very
  familiar with \textbf{lattice}
\item
  Note: It is possible to manipulate the \textbf{plot()} options to get
  a plot similar to ggplot (and maybe even a better one), but that would
  require a lot of extra coding!!
\item
  The ggplot \emph{syntax} also makes plotting more \textbf{structured}
  and easier to update
\end{itemize}

\hypertarget{ggplot-syntax}{%
\subsection{ggplot syntax}\label{ggplot-syntax}}

ggplot(diamonds, aes(x=carat,y=price)) + geom\_point()

\emph{The first component (before the ``+'') calls the ggplot function,
and the data with x-y varibles. } The second component (after the first
``+'') tells ggplot what type of plot you want, e.g.,
geom\_bar/geom\_hist/geom\_boxplot * Possible to add other lines for
more customization on the plot, e.g., title, label for the axes, etc. **
\url{https://rkabacoff.github.io/datavis/} a whole book on ggplot2
customization!!

\hypertarget{geom_boxplot}{%
\subsection{geom\_boxplot}\label{geom_boxplot}}

Very easy to switch to a boxplot. We can use \textbf{geom\_boxplot} to
create boxplots when one variable is continuous and the other is a
factor.

\begin{Shaded}
\begin{Highlighting}[]
\FunctionTok{ggplot}\NormalTok{(diamonds, }\FunctionTok{aes}\NormalTok{(}\AttributeTok{x=}\NormalTok{cut,}\AttributeTok{y=}\NormalTok{price)) }\SpecialCharTok{+} \FunctionTok{geom\_boxplot}\NormalTok{()}
\end{Highlighting}
\end{Shaded}

\begin{center}\includegraphics{ggplot_class_practice_files/figure-latex/unnamed-chunk-6-1} \end{center}

\hypertarget{changing-the-aesthetics}{%
\subsection{Changing the aesthetics}\label{changing-the-aesthetics}}

You can control the aesthetics of each layer, e.g.~\emph{colour},
\emph{size}, \emph{shape}, \emph{alpha} (opacity) etc.
\url{https://ggplot2.tidyverse.org/reference/geom_point.html}

\begin{Shaded}
\begin{Highlighting}[]
\FunctionTok{ggplot}\NormalTok{(diamonds, }\FunctionTok{aes}\NormalTok{(carat, price)) }\SpecialCharTok{+} \FunctionTok{geom\_point}\NormalTok{(}\AttributeTok{col =} \StringTok{"blue"}\NormalTok{)}
\end{Highlighting}
\end{Shaded}

\begin{center}\includegraphics{ggplot_class_practice_files/figure-latex/unnamed-chunk-7-1} \end{center}

\hypertarget{a-few-more-examples}{%
\subsection{A few more examples}\label{a-few-more-examples}}

Changing the alpha level

\begin{Shaded}
\begin{Highlighting}[]
\FunctionTok{ggplot}\NormalTok{(diamonds, }\FunctionTok{aes}\NormalTok{(}\AttributeTok{x=}\NormalTok{carat,}\AttributeTok{y=}\NormalTok{price)) }\SpecialCharTok{+} \FunctionTok{geom\_point}\NormalTok{(}\AttributeTok{alpha =} \FloatTok{0.2}\NormalTok{)}
\end{Highlighting}
\end{Shaded}

\begin{center}\includegraphics{ggplot_class_practice_files/figure-latex/unnamed-chunk-8-1} \end{center}

\hypertarget{a-few-more-examples-1}{%
\subsection{A few more examples}\label{a-few-more-examples-1}}

Changing the point size

\begin{Shaded}
\begin{Highlighting}[]
\FunctionTok{ggplot}\NormalTok{(diamonds, }\FunctionTok{aes}\NormalTok{(}\AttributeTok{x=}\NormalTok{carat,}\AttributeTok{y=}\NormalTok{price)) }\SpecialCharTok{+} \FunctionTok{geom\_point}\NormalTok{(}\AttributeTok{size =} \FloatTok{0.2}\NormalTok{)}
\end{Highlighting}
\end{Shaded}

\begin{center}\includegraphics{ggplot_class_practice_files/figure-latex/unnamed-chunk-9-1} \end{center}

\hypertarget{a-few-more-examples-2}{%
\subsection{A few more examples}\label{a-few-more-examples-2}}

Changing the shape and the point size

\begin{Shaded}
\begin{Highlighting}[]
\FunctionTok{ggplot}\NormalTok{(diamonds, }\FunctionTok{aes}\NormalTok{(}\AttributeTok{x=}\NormalTok{carat,}\AttributeTok{y=}\NormalTok{price)) }\SpecialCharTok{+} \FunctionTok{geom\_point}\NormalTok{(}\AttributeTok{shape =} \DecValTok{2}\NormalTok{,}\AttributeTok{size=}\FloatTok{0.4}\NormalTok{)}
\end{Highlighting}
\end{Shaded}

\begin{center}\includegraphics{ggplot_class_practice_files/figure-latex/unnamed-chunk-10-1} \end{center}

\hypertarget{combining-layers}{%
\subsection{Combining layers}\label{combining-layers}}

The real power of \textbf{ggplot} is its ability to combine layers

\begin{Shaded}
\begin{Highlighting}[]
\FunctionTok{ggplot}\NormalTok{(diamonds, }\FunctionTok{aes}\NormalTok{(}\AttributeTok{x=}\NormalTok{carat,}\AttributeTok{y=}\NormalTok{price)) }\SpecialCharTok{+} \FunctionTok{geom\_point}\NormalTok{(}\AttributeTok{size =} \FloatTok{0.2}\NormalTok{) }\SpecialCharTok{+}
  \FunctionTok{geom\_smooth}\NormalTok{()}
\end{Highlighting}
\end{Shaded}

\begin{center}\includegraphics{ggplot_class_practice_files/figure-latex/unnamed-chunk-11-1} \end{center}

\hypertarget{transformations}{%
\subsection{Transformations}\label{transformations}}

In this case (and many other situations) a log transformation may allow
for the relationships between variables to be clearer. Can use
\textbf{coord\_trans()}

\begin{Shaded}
\begin{Highlighting}[]
\FunctionTok{ggplot}\NormalTok{(diamonds, }\FunctionTok{aes}\NormalTok{(carat, price)) }\SpecialCharTok{+} \FunctionTok{geom\_point}\NormalTok{(}\AttributeTok{size =} \FloatTok{0.5}\NormalTok{) }\SpecialCharTok{+}
\FunctionTok{coord\_trans}\NormalTok{(}\AttributeTok{x =} \StringTok{"log10"}\NormalTok{, }\AttributeTok{y =} \StringTok{"log10"}\NormalTok{)}
\end{Highlighting}
\end{Shaded}

\begin{center}\includegraphics{ggplot_class_practice_files/figure-latex/unnamed-chunk-12-1} \end{center}

\hypertarget{adding-information-for-a-third-variable}{%
\subsection{Adding information for a third
variable}\label{adding-information-for-a-third-variable}}

We can color by a \emph{factor} variable (not that it's useful for this
example!)

\begin{Shaded}
\begin{Highlighting}[]
\FunctionTok{ggplot}\NormalTok{(diamonds, }\FunctionTok{aes}\NormalTok{(carat, price, }\AttributeTok{colour=}\NormalTok{color)) }\SpecialCharTok{+} \FunctionTok{geom\_point}\NormalTok{() }\SpecialCharTok{+} 
    \FunctionTok{coord\_trans}\NormalTok{(}\AttributeTok{x =} \StringTok{"log10"}\NormalTok{, }\AttributeTok{y =} \StringTok{"log10"}\NormalTok{)}
\end{Highlighting}
\end{Shaded}

\begin{center}\includegraphics{ggplot_class_practice_files/figure-latex/unnamed-chunk-13-1} \end{center}

Can also color by a \emph{continuous} variable (not really useful for
this example too, but here it is so you are familiar with the syntax:)

\begin{Shaded}
\begin{Highlighting}[]
\FunctionTok{ggplot}\NormalTok{(diamonds, }\FunctionTok{aes}\NormalTok{(carat, price, }\AttributeTok{colour=}\NormalTok{depth)) }\SpecialCharTok{+} \FunctionTok{geom\_point}\NormalTok{() }\SpecialCharTok{+} 
    \FunctionTok{coord\_trans}\NormalTok{(}\AttributeTok{x =} \StringTok{"log10"}\NormalTok{, }\AttributeTok{y =} \StringTok{"log10"}\NormalTok{)}
\end{Highlighting}
\end{Shaded}

\begin{center}\includegraphics{ggplot_class_practice_files/figure-latex/unnamed-chunk-14-1} \end{center}

\hypertarget{conditional-plots}{%
\subsection{Conditional plots:}\label{conditional-plots}}

In some cases, it may be more useful to get separate plots for each
category of the third variable, to understand conditional relationships

\begin{Shaded}
\begin{Highlighting}[]
\FunctionTok{ggplot}\NormalTok{(diamonds, }\FunctionTok{aes}\NormalTok{(carat, price)) }\SpecialCharTok{+} \FunctionTok{geom\_point}\NormalTok{() }\SpecialCharTok{+}
  \FunctionTok{facet\_wrap}\NormalTok{(}\SpecialCharTok{\textasciitilde{}}\NormalTok{color, }\AttributeTok{ncol=}\DecValTok{4}\NormalTok{)}
\end{Highlighting}
\end{Shaded}

\begin{center}\includegraphics{ggplot_class_practice_files/figure-latex/unnamed-chunk-15-1} \end{center}

Alternatively, you can use \textbf{facet\_grid}, which also allows more
than 1 conditioning variable (tables of plots)

\begin{Shaded}
\begin{Highlighting}[]
\FunctionTok{ggplot}\NormalTok{(diamonds, }\FunctionTok{aes}\NormalTok{(carat, price)) }\SpecialCharTok{+} \FunctionTok{geom\_point}\NormalTok{() }\SpecialCharTok{+}
 \FunctionTok{facet\_grid}\NormalTok{(}\SpecialCharTok{\textasciitilde{}}\NormalTok{color, }\AttributeTok{labeller=}\NormalTok{label\_both)}
\end{Highlighting}
\end{Shaded}

\begin{center}\includegraphics{ggplot_class_practice_files/figure-latex/unnamed-chunk-16-1} \end{center}

\hypertarget{a-final-note-about-syntax}{%
\subsection{A final note about syntax}\label{a-final-note-about-syntax}}

There are actually many ways to get the same plot! The following
commands will produce the same plot:

\begin{itemize}
\tightlist
\item
  ggplot(diamonds, aes(price, carat)) + geom\_point()
\item
  ggplot() + geom\_point(aes(price, carat), diamonds)
\item
  ggplot(diamonds) + geom\_point(aes(price, carat))
\item
  ggplot(diamonds, aes(price)) + geom\_point(aes(y = carat))
\item
  ggplot(diamonds, aes(y=carat)) + geom\_point(aes(x = price))
\end{itemize}

\hypertarget{a-final-summary-of-syntax}{%
\subsection{A final summary of
syntax:}\label{a-final-summary-of-syntax}}

\begin{Shaded}
\begin{Highlighting}[]
\FunctionTok{ggplot}\NormalTok{(diamonds) }\SpecialCharTok{+} \FunctionTok{geom\_point}\NormalTok{(}\FunctionTok{aes}\NormalTok{(price, carat))}
\end{Highlighting}
\end{Shaded}

\begin{center}\includegraphics{ggplot_class_practice_files/figure-latex/unnamed-chunk-17-1} \end{center}

\hypertarget{histograms}{%
\subsection{Histograms}\label{histograms}}

Let's make a histogram.

\begin{Shaded}
\begin{Highlighting}[]
\FunctionTok{ggplot}\NormalTok{(diamonds, }\FunctionTok{aes}\NormalTok{(depth)) }\SpecialCharTok{+} \FunctionTok{geom\_histogram}\NormalTok{()}
\end{Highlighting}
\end{Shaded}

\begin{center}\includegraphics{ggplot_class_practice_files/figure-latex/unnamed-chunk-18-1} \end{center}

Notice the difference in the \textbf{aes} call; \textbf{boxplot} is
really designed for multiple categories!

\hypertarget{histograms-1}{%
\subsection{Histograms}\label{histograms-1}}

Tthe \emph{default options} in \textbf{histogram} may not be sensible,
and you often need to adjust the \textbf{binwidth} and \textbf{xlim}

\begin{Shaded}
\begin{Highlighting}[]
\FunctionTok{ggplot}\NormalTok{(diamonds, }\FunctionTok{aes}\NormalTok{(depth)) }\SpecialCharTok{+} \FunctionTok{geom\_histogram}\NormalTok{(}\AttributeTok{binwidth=}\FloatTok{0.2}\NormalTok{) }\SpecialCharTok{+} \FunctionTok{xlim}\NormalTok{(}\DecValTok{56}\NormalTok{,}\DecValTok{67}\NormalTok{)}
\end{Highlighting}
\end{Shaded}

\begin{center}\includegraphics{ggplot_class_practice_files/figure-latex/unnamed-chunk-19-1} \end{center}

\hypertarget{boxplots-with-multiple-categories}{%
\subsection{Boxplots with multiple
categories}\label{boxplots-with-multiple-categories}}

A better use of \textbf{boxplot} is when we want to compare
distributions of a quantitative variable across categories of a factor
variable, as previously discussed

\begin{Shaded}
\begin{Highlighting}[]
\FunctionTok{ggplot}\NormalTok{(diamonds, }\FunctionTok{aes}\NormalTok{(cut, depth)) }\SpecialCharTok{+} \FunctionTok{geom\_boxplot}\NormalTok{()}
\end{Highlighting}
\end{Shaded}

\begin{center}\includegraphics{ggplot_class_practice_files/figure-latex/unnamed-chunk-20-1} \end{center}

\hypertarget{histograms-with-multiple-categories}{%
\subsection{Histograms with multiple
categories}\label{histograms-with-multiple-categories}}

We can also get multiple histograms, though we need to either display
them separately (less useful when comparing)

\begin{Shaded}
\begin{Highlighting}[]
\FunctionTok{ggplot}\NormalTok{(diamonds, }\FunctionTok{aes}\NormalTok{(depth)) }\SpecialCharTok{+} \FunctionTok{geom\_histogram}\NormalTok{(}\AttributeTok{binwidth =} \FloatTok{0.2}\NormalTok{) }\SpecialCharTok{+} 
    \FunctionTok{facet\_wrap}\NormalTok{(}\SpecialCharTok{\textasciitilde{}}\NormalTok{cut) }\SpecialCharTok{+} \FunctionTok{xlim}\NormalTok{(}\DecValTok{56}\NormalTok{, }\DecValTok{67}\NormalTok{)}
\end{Highlighting}
\end{Shaded}

\begin{center}\includegraphics{ggplot_class_practice_files/figure-latex/unnamed-chunk-21-1} \end{center}

\hypertarget{overlaying-histograms}{%
\subsection{Overlaying Histograms}\label{overlaying-histograms}}

Or, you can overlay the historgrams

\begin{Shaded}
\begin{Highlighting}[]
\FunctionTok{ggplot}\NormalTok{(diamonds, }\FunctionTok{aes}\NormalTok{(depth, }\AttributeTok{fill=}\NormalTok{cut)) }\SpecialCharTok{+} 
    \FunctionTok{geom\_histogram}\NormalTok{(}\AttributeTok{binwidth=}\FloatTok{0.2}\NormalTok{,}\AttributeTok{colour=}\StringTok{"grey50"}\NormalTok{,}\AttributeTok{alpha=}\NormalTok{.}\DecValTok{4}\NormalTok{,}\AttributeTok{position=}\StringTok{"identity"}\NormalTok{) }\SpecialCharTok{+} \FunctionTok{xlim}\NormalTok{(}\DecValTok{56}\NormalTok{,}\DecValTok{67}\NormalTok{)}
\end{Highlighting}
\end{Shaded}

\begin{center}\includegraphics{ggplot_class_practice_files/figure-latex/unnamed-chunk-22-1} \end{center}

\hypertarget{cheat-sheet}{%
\subsection{Cheat sheet}\label{cheat-sheet}}

We are covering only a few of the many plot types that can be greated
with the \textbf{ggplot2} package.

For a more comprehensive view of ggplot2, take a look at the ggplot2
\href{https://r4ds.had.co.nz/data-visualisation.html}{Cheat sheet}

\hypertarget{maps}{%
\subsection{Maps}\label{maps}}

\hypertarget{load-united-states-state-map-data}{%
\subsubsection{load United States state map
data}\label{load-united-states-state-map-data}}

\begin{Shaded}
\begin{Highlighting}[]
\CommentTok{\#install.packages(\textquotesingle{}maps\textquotesingle{}) \# you only need to do this once. maps package includes various maps that we can use.}
\CommentTok{\#install.packages(\textquotesingle{}sf\textquotesingle{}) \# you only need to do this once}
\FunctionTok{library}\NormalTok{(maps)     }\CommentTok{\# Provides latitude and longitude data for various maps}
\FunctionTok{library}\NormalTok{(sf)}
\CommentTok{\# read the state population data}
\NormalTok{MainStates }\OtherTok{\textless{}{-}} \FunctionTok{map\_data}\NormalTok{(}\StringTok{"state"}\NormalTok{)}

\CommentTok{\#plot all states with ggplot2, using black borders and light blue fill}
\FunctionTok{ggplot}\NormalTok{() }\SpecialCharTok{+} 
  \FunctionTok{geom\_polygon}\NormalTok{( }\AttributeTok{data=}\NormalTok{MainStates, }\FunctionTok{aes}\NormalTok{(}\AttributeTok{x=}\NormalTok{long, }\AttributeTok{y=}\NormalTok{lat, }\AttributeTok{group=}\NormalTok{group),}
                \AttributeTok{color=}\StringTok{"black"}\NormalTok{, }\AttributeTok{fill=}\StringTok{"lightblue"}\NormalTok{ ) }\SpecialCharTok{+}
                \FunctionTok{coord\_sf}\NormalTok{(}\AttributeTok{crs =} \FunctionTok{st\_crs}\NormalTok{(}\DecValTok{4326}\NormalTok{)) }\CommentTok{\# projection}
\end{Highlighting}
\end{Shaded}

\begin{center}\includegraphics{ggplot_class_practice_files/figure-latex/unnamed-chunk-23-1} \end{center}

\end{document}
